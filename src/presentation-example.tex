\documentclass[ignorenonframetext,]{beamer}
\setbeamertemplate{caption}[numbered]
\setbeamertemplate{caption label separator}{: }
\setbeamercolor{caption name}{fg=normal text.fg}
\usepackage{lmodern}
\usepackage{amssymb,amsmath}
\usepackage{ifxetex,ifluatex}
\usepackage{fixltx2e} % provides \textsubscript
\ifnum 0\ifxetex 1\fi\ifluatex 1\fi=0 % if pdftex
  \usepackage[T1]{fontenc}
  \usepackage[utf8]{inputenc}
\else % if luatex or xelatex
  \ifxetex
    \usepackage{mathspec}
  \else
    \usepackage{fontspec}
  \fi
  \defaultfontfeatures{Ligatures=TeX,Scale=MatchLowercase}
  \newcommand{\euro}{€}
\fi
% use upquote if available, for straight quotes in verbatim environments
\IfFileExists{upquote.sty}{\usepackage{upquote}}{}
% use microtype if available
\IfFileExists{microtype.sty}{%
\usepackage{microtype}
\UseMicrotypeSet[protrusion]{basicmath} % disable protrusion for tt fonts
}{}
\usepackage{longtable,booktabs}
\usepackage[font={footnotesize}]{caption}
% These lines are needed to make table captions work with longtable:
\makeatletter
\def\fnum@table{\tablename~\thetable}
\makeatother

% Comment these out if you don't want a slide with just the
% part/section/subsection/subsubsection title:
\AtBeginPart{
  \let\insertpartnumber\relax
  \let\partname\relax
  \frame{\partpage}
}
\AtBeginSection{
  \let\insertsectionnumber\relax
  \let\sectionname\relax
  \frame{\sectionpage}
}
\AtBeginSubsection{
  \let\insertsubsectionnumber\relax
  \let\subsectionname\relax
  \frame{\subsectionpage}
}

\setlength{\emergencystretch}{3em}  % prevent overfull lines
\providecommand{\tightlist}{%
  \setlength{\itemsep}{0pt}\setlength{\parskip}{0pt}}
\setcounter{secnumdepth}{0}

\date{}

%% Here's everything I added.
%%--------------------------
\usepackage{graphicx}
\usepackage{rotating}
\usepackage{hyperref}
% \usepackage[font={footnotesize}]{caption}
\usepackage[normalem]{ulem}
\usepackage{etoolbox}
\usepackage{wasysym}


% Get rid of navigation symbols.
%-------------------------------
\setbeamertemplate{navigation symbols}{}

% Optional institute tags and titlegraphic.
% Do feel free to change the titlegraphic if you don't want it as a Markdown field.
%----------------------------------------------------------------------------------


% \titlegraphic{\includegraphics[width=0.3\paperwidth]{\string~/Dropbox/teaching/clemson-academic.png}} % <-- if you want to know what this looks like without it as a Markdown field.
% -----------------------------------------------------------------------------------------------------


% Some additional title page adjustments.
%----------------------------------------
% \setbeamertemplate{title page}[empty]

\defbeamertemplate{background canvas}{body}{%
  \includegraphics[width=\paperwidth,height=\paperheight]{style/body_page.png}
}

\BeforeBeginEnvironment{frame}{%
  \setbeamertemplate{background canvas}[body]%
}

\makeatletter
\define@key{beamerframe}{body}[true]{%
  \setbeamertemplate{background canvas}[body]%
}
\makeatother
%\date{}
\setbeamerfont{subtitle}{size=\small}
\setbeamerfont{footnote}{size=\tiny}
\setbeamerfont*{title}{size=\huge}
% \setbeamercovered{transparent}

\setbeamercolor{itemize item}{fg=orange}
\setbeamercolor{itemize subitem}{fg=orange}
\setbeamercolor{itemize subsubitem}{fg=orange}

\setbeamertemplate{itemize item}[circle]
\setbeamertemplate{itemize subitem}[square]
\setbeamertemplate{itemize subsubitem}[triangle]

\definecolor{cidacsred}{HTML}{761514}
\setbeamercolor{frametitle}{fg=cidacsred}
\setbeamercolor{local structure}{fg=cidacsred}
% \setbeamercolor{footline}{fg=orange!50, bg=white}
\setbeamercolor{titlelike}{fg=cidacsred}
\setbeamercolor{footnote}{bg=white}
\setbeamercolor{normal text}{fg=black}
\setbeamercolor{block title}{fg=orange!70!black, bg=white}

\let\Tiny=\tiny


% Sections and subsections should not get their own damn slide.
%--------------------------------------------------------------
\AtBeginPart{}
\AtBeginSection{}
\AtBeginSubsection{}
\AtBeginSubsubsection{}

% Suppress some of Markdown's weird default vertical spacing.
%------------------------------------------------------------
\setlength{\emergencystretch}{0em}  % prevent overfull lines
\setlength{\parskip}{0pt}


% Progression bar at the bottom
%--------------------------------------------------
\makeatletter
\addtobeamertemplate{footline}{%
  \color{cidacsred}% to color the progressbar
  \hspace*{-\beamer@leftmargin}%
  \rule{\beamer@leftmargin}{2pt}%
  \rlap{\rule{\dimexpr
      \beamer@startpageofframe\dimexpr
      \beamer@rightmargin+\textwidth\relax/\beamer@endpageofdocument}{1pt}}
  % next 'empty' line is mandatory!
  % \vspace{0\baselineskip}
  {}
}


% \AtBeginDocument{%
%   \letcs\oig{@orig\string\includegraphics}%
%   \renewcommand<>\includegraphics[2][]{%
%     \only#3{%
%       {\centering\oig[{#1}]{#2}\par}%
%     }%
%   }%
% }

% I think I've moved to xelatex now. Here's some stuff for that.
% --------------------------------------------------------------
% I could customize/generalize this more but the truth is it works for my circumstances.

\ifxetex
\setbeamerfont{title}{family=\fontspec{serif}}
\setbeamerfont{frametitle}{family=\fontspec{serif}}
% \usepackage[font=small,skip=0pt]{caption}
 \else
 \fi


% Okay, and begin the actual document...

\begin{document}




\begin{frame}
\begin{longtable}[]{@{}l@{}}
\toprule
\endhead
title: ``Work Schedule'' \\
author: Me\^{}{[}Post Doc \textbar{} Center of Data and Knowledge
Integration for Health (CIDACS) \\
\textbar{} Salvador \textbar{} Brasil{]} \\
date: ``\today'' \\
output: \\
beamer\_presentation: \\
template: style/template.tex \\
ioslides\_presentation: default \\
slidy\_presentation: default \\
institute: Center of Data and Knowledge Integration for Health
(CIDACS) \\
titlegraphic: ./style/title.png \\
csl: style/chicago-fullnote-bibliography.csl \\
suppress-bibliography: yes \\
bibliography: /home/luis/Documents/library.bib \\
lang: en-AU \\
subtitle: Project subtitle \\
classoption: aspectratio=169 \\
header-includes: \textbar{} \\
 \\
\embeddedfile{project}{presentation-example.Rmd} \\
 \\
\bottomrule
\end{longtable}

\begin{block}{First slide}
\protect\hypertarget{first-slide}{}
\begin{itemize}
\tightlist
\item
  Occurs in all ages

  \begin{itemize}
  \tightlist
  \item
    More common in the elderly
  \end{itemize}
\item
  Example of citation (\textbf{machadoEffectsCOVID19Anxiety2020?}).
\end{itemize}
\end{block}
\end{frame}

\begin{frame}{Sections are not supported}
\protect\hypertarget{sections-are-not-supported}{}
\begin{block}{Data set - CIDACS}
\protect\hypertarget{data-set---cidacs}{}
\begin{columns}[T]
\begin{column}{0.5\textwidth}
\begin{block}{A block}
\protect\hypertarget{a-block}{}
\begin{itemize}
\tightlist
\item
  Column example
\end{itemize}
\end{block}
\end{column}

\begin{column}{0.5\textwidth}
\tiny

Dataset coverage from 2001 to 2017
\end{column}
\end{columns}
\end{block}

\begin{block}{Work schedule}
\protect\hypertarget{work-schedule}{}
\end{block}
\end{frame}

\begin{frame}{Acknowledgements}
\protect\hypertarget{acknowledgements}{}
\begin{columns}[T]
\begin{column}{0.5\textwidth}
\begin{block}{Team}
\protect\hypertarget{team}{}
\begin{itemize}
\item
  Prof X
\item
  Prof Y
\item
  grant no 123
\end{itemize}
\end{block}
\end{column}

\begin{column}{0.5\textwidth}
\begin{block}{Contact}
\protect\hypertarget{contact}{}
\begin{center}\includegraphics[width=0.8\linewidth]{../graphs/qr-twitter-1} \end{center}
\end{block}
\end{column}
\end{columns}
\end{frame}


\end{document}


% \embeddedfile{projeto}{../../presentation.zip}
